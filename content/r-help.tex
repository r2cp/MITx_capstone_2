\subsection {ATE abd Selection Bias} 
When we observe a group of subjects, some of whom have been treated and some not, we calculate
\begin {equation} \begin {split}
& \EE(Y_i(1) \g  W_i = 1) - \EE(Y_i(0) \g  W_i = 0) \iff \\
& [ \EE(Y_i(1) \g  W_i = 1) -  \EE(Y_i(0) \g  W_i = 1) ] +  \\
& \qquad \EE(Y_i(0) \g  W_i = 1) -  \EE(Y_i(0) \g  W_i = 0) ] \iff  \\
& \text {Average Treatment Effect} + \text { Selection Bias} \\
\end {split} \end {equation}

\subsection {Fisher exact test in R}
\begin{lstlisting}
library(perm)
perms <- chooseMatrix(6, 3)
A <- matrix(c(38.2, 37.1, 37.6, 36.4, 37.3, 36), nrow=6, ncol=1, byrow=TRUE)
is.treatment <- c(1, 1, 1, 0, 0, 0)

n_treatment <- sum(is.treatment)
n_control <- length(A) - sum(is.treatment)
treatment_avg <- (1/n_treatment) * perms %*% A
control_avg <- (1/3) * (1-perms) %*% A
test_statistic <- abs(treatment_avg - control_avg)

rownumber <- apply(apply(perms, 1, 
    function(x) (x == is.treatment)), 2, sum)
rownumber <- (rownumber == length(A))
observed_test <- test_statistic[rownumber == TRUE]

larger_than_observed <- (test_statistic >= observed_test)
sum(larger_than_observed) / length(test_statistic)
\end{lstlisting}


\subsection {Fixed effects model} \label {r: fem}
G(factor(admin)) creates a dummy variable, one per region, and includes the dummy variable
in the regression
\begin{lstlisting}[language=R]
library("lfe")
model2 <- felm(sex ~ teasown + post + teapost + 
G(factor(admin)), data = qiandata)
summary(model2)
\end{lstlisting}

% \subsection {Instrument variable regression}
% To use the ivreg() function directly one needs to setup the call as
% \begin{lstlisting}[language=R]
% ivreg (y ~ exogenous_vars + endogenous_var | 
% exogenous_vars + instrument_var)
% \end{lstlisting}


\subsection{Neyman analysis}
\begin{lstlisting}
data <- read.csv("data_myData.csv")
treatment <- data$T == 1

treatment_mean <- mean(data$Y[treatment], na.rm=TRUE)
control_mean <- mean(data$Y[!treatment], na.rm=TRUE)
ate <- treatment_mean - control_mean

Nc <- sum(!treatment)
Nt <- sum(treatment)
Sc2 <- (1/(Nc-1)) * sum( (data$Y[!treatment] - control_mean)^2 )
St2 <- (1/(Nt-1)) * sum( (data$Y[treatment] - treatment_mean)^2 )

Vneyman <- (Sc2/Nc + St2/Nt)

# N > 30
ub <- ate + 1.96 * sqrt(Vneyman)
lb <- ate - 1.96 * sqrt(Vneyman)

# t-distribution analysis
g <- ( Vneyman^2 / ( (St2/Nt)^2/(Nt+1) + (Sc2/Nc)^2/(Nc+1) ) ) - 2
ub <- ate + qt(0.975, g) * sqrt(Vneyman)
lb <- ate - qt(0.975, g) * sqrt(Vneyman)
\end{lstlisting}

\subsection{Regression commands}
\begin{lstlisting}
hprices <- read.csv("data_House_Prices_and_Crime_1.csv")

str(hprices)
summary(hprices)
subset(hprices, index_nsa == 54.29)

model1 <- lm(index_nsa ~ Homicides + Robberies + Assaults, data=hprices)
summary(model1)
confint(model1)
\end{lstlisting}

\subsection{F-test, restricted model}
\begin{lstlisting}
model_unrest <- lm(index_nsa ~ Homicides + Robberies + Assaults, data=hprices)
anova_unrest <- anova(model_unrest)
model_rest <- lm(index_nsa ~ I(Homicides-Assaults) + I(Robberies-Assaults), data=hprices)
anova_rest <- anova(model_rest)

# F statistic
r <- 1
k <- 3
ssr_u <- anova_unrest$`Sum Sq`[4]
ssr_r <- anova_rest$`Sum Sq`[length(anova_rest$`Sum Sq`)]
statistic_test <- (((ssr_r - ssr_u)/r) / ((ssr_u) / anova_unrest$Df[4]))
pvalue <- df(statistic_test, r, anova_unrest$Df[4])
\end{lstlisting}

\subsection{QQ-plots}
The R base functions \lstinline{qqnorm()} and \lstinline{qqline()} can be used to produce quantile-quantile plots:
\begin{lstlisting}
data <- runif(1000)
qqnorm(data)
qqline(data)
\end{lstlisting}

\subsection{Density plots}
\begin{lstlisting}
data <- runif(1000)
d <- density(data, bw = 0.01)
plot(d)
\end{lstlisting}


\subsection{Difference in difference model}
\begin{lstlisting}
manufacturing <- read.csv("data_manufacturing.csv")

# Cleaning
library(tidyverse)
manu <- manufacturing %>% 
filter((year == 1987 | year == 1988) & !is.na(scrap)) %>%
select(year, fcode, scrap, grant)

treated <- manu %>%
    filter(grant == 1) %>%
    select(fcode)
treated_firms <- treated$fcode

# Add column 'treated'
manu <- manu %>%
    mutate(treated = 1*(fcode %in% treated_firms))

# Average treatment and control 
did_results <- manu %>%
    group_by(year, treated) %>%
    summarize(promedio = mean(scrap), n = n())
did_results

# DiD model
did_model <- lm(scrap ~ treated + I(year==1988) + I(treated*(year==1988)), data=manu)
\end{lstlisting}


\subsection{R distribution functions}
The functions for the density/mass function, cumulative distribution function, quantile function and random variate generation are named in the form \lstinline{dxxx}, \lstinline{pxxx}, \lstinline{qxxx} and \lstinline{rxxx} respectively.
\begin{itemize}
\item For the beta distribution see \lstinline{dbeta}.
\item For the binomial (including Bernoulli) distribution see \lstinline{dbinom}.
\item For the Cauchy distribution see \lstinline{dcauchy}.
\item For the chi-squared distribution see \lstinline{dchisq}.
\item For the exponential distribution see \lstinline{dexp}.
\item For the F distribution see \lstinline{df}.
\item For the gamma distribution see \lstinline{dgamma}.
\item For the geometric distribution see \lstinline{dgeom}. (This is also a special case of the negative binomial.)
\item For the hypergeometric distribution see \lstinline{dhyper}.
\item For the log-normal distribution see \lstinline{dlnorm}.
\item For the multinomial distribution see \lstinline{dmultinom}.
\item For the negative binomial distribution see \lstinline{dnbinom}.
\item For the normal distribution see \lstinline{dnorm}.
\item For the Poisson distribution see \lstinline{dpois}.
\item For the Student's t distribution see \lstinline{dt}.
\item For the uniform distribution see \lstinline{dunif}.
\item For the Weibull distribution see \lstinline{dweibull}.
\end{itemize}

